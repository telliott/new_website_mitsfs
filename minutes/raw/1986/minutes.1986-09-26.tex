\documentclass[12pt]{article}
\setlength{\topmargin}{-0.5in}
\setlength{\oddsidemargin}{0.0in}
\setlength{\evensidemargin}{0.0in}
\setlength{\textheight}{9in}
\setlength{\textwidth}{6.5in}

\newcommand{\bing}{{\bf BING!} }
\newcommand{\goto}[1]{\bing \vskip 12pt \centerline{{\em{#1}}}}

\begin{document}

\begin{center}

MIT Science Fiction Society 

84 Massachusetts Avenue

Cambridge, MA 02139

\vspace{12pt}

MITSFS Meeting Minutes 

Friday, September 26, 1986

\end{center}
 
\vspace{18pt}

\setlength{\parskip}{6pt}

\noindent
MITSFS meeting called to order, 1700 SST,
Susan Tucker, Skinner, presiding.

Minutes of the previous meeting read.

(JME) Move to approve the minutes as not nearly as exciting as my surgery.

(DSK) Second.

Motion passes everyone+"Hi, Bo!"-1-1 +Spehn +1 vote for none of the above.

\goto{Committee Reports}

(JME) Jourcomm: write for TZ!!! Also, anyone who adds books for review to my list on the board should leave a note for me so I can keep track of them myself.

(RJG) Pseudo-Famecomm: the pictures and story of the MITSFS that Newsweek On Campus did during spring term didn't appear in this issue, perhaps in the next.

(JME) Semi-Famecomm: in last Sunday's Glob magazine there appeared a horrible article on local science fiction authors that fortunately did not mention the MITSFS. To give you an idea: the writer covered a "Bash," the local annual STrek con, and, while admitting that it was not a regular SF con, they did not manage to mention what made the distinction. They talked to Grace Lee Whitney, who has some weird ideas about mystic connections and STrek, mentioned "teenage girls with brain-dead eyes," and referred to the fans in the dealer's room as "fevered lemmings." After this, they commented that "some conventions are so paranoid they won't even issue press passes." They managed to get most of the details about Joe Haldeman wrong, so they probably got the details about the other authors wrong too.

(DSK) Sitcomm: Raiders of the Lost Ark on ABC on Sunday night, making its TV premiere with only 24 seconds cut.

(JME) Sitcomm sub 2: there's a new half-hour comedy series on Friday nights called Sledge Hammer!. It's a parody of shows like Miami Vice and Hunter and features a main character whose favorite line is "Trust me, I know what I'm doing." The premiere was pretty funny.

(Some flaming about Sledge Hammer! versus Police Squad.)

(SSDT) Pseudo-Moocomm: the Herald reports grumblings about Back to the Future II which might concern the main characters traveling to the future to visit their kids.

(JME) Pseudo-Moocomm sub 2: a movie called Peggy Sue Got Married is supposedly premiering tonight. Apparently it concerns a woman whose consciousness travels back in time and winds up inside her teenage body. It's a Francis Ford Coppola movie.

(LAK) Moocomm: The Boy Who Could Fly opens today. It looks kind of sappy but kind of cute too. It's not clear whether or not it is truly fantasy.

(SSDT) Early reviews are very mixed: the Glob gave it four stars, the Herald gave it one.

(DSK) Sitcomm sub 3: Hill Street Blues season opener is next Thursday.

(JME) Sitcomm sub 4: Alf has premiered, if anyone cares.

(SLP) The alien was the best actor of the bunch.

(AA) Sitcomm sub 5: Twilight Zone is showing as one of its stories Ted Sturgeon's "A Saucer Full of Loneliness."

(JME) Infocomm: Infocom has two new games out- "Trinity," a fantasy game having to do with nuclear weaponry, and it's very well-written; and "Leather Goddesses of Phobos," due out shortly. This is a salute to old pulp magazine SF and comes in three naughtiness levels. It will also ask you if you are male or female and modify the game accordingly. AND it comes with a Scratch N' Sniff card. This is an anti-piracy move- it's tough to Xerox a Scratch N' Sniff card.

(CH) Sitcomm sub 6: I caught Starman and I'm going to try and catch him again.

(There are many cries of "What did you do with him?" to which Connie replies "I have some ideas." There is some discussion of whether Robert Hayes or Jeff Bridges would be more worth catching, and Rob suggests that Kathleen Turner would make a good Starman.)

(CH) Sitcomm sub 6, take two: the premiere concerned how he gets back to Earth and takes over the job of a famous photographer. I enjoyed it. There was a small chase but no shots were fired and only one helicopter crash off screen; they seem to be making it very nonviolent which means it'll probably get killed in a Friday-at-10 timeslot.

(JME) Sitcomm sub 7: various cliffhangers have been resolved and the Dallas "Bobby in the shower" scene gets resolved tonight. The Phoenix's TV critic suggested that it was a parallel universe. On Cheers, Sam proposed to Diane, she rejected it, then changed her mind later and he said "Fuck off."

\goto{Old Business}

(JME) I did indeed have my operation last Friday at approximately this time. Sue Hagadorn never did bring me that ice cream.

OBA: um, us, uv, ur.

\goto{New Business}

(JME) Minicult: a woman is suing Disney World because a man in a Mickey Mouse suit beat up her four-year-old when the child grabbed him by the tail.

(Jim Rauen) Minicult: in Tuesday's "Ask the Glob" someone asked if blasphemy was still illegal in Massachusetts. The answer was yes; blasphemy carries a fine of up to 500 dollars or a year in jail. Another article I saw talked about how people were stealing antique weather vanes off of public buildings. The people who are responsible for this could be prosecuted under the blasphemy laws, because all weather vanes on public buildings in Massachusetts have "In God We Trust" engraved on them, and stealing one would be taking the name of the Lord in vane.

(There is a resounding cry of "ALBANIAN MOTION!!!" from all sentient beings in the room and many of the only semi-sentient ones.)

(JME) Minicult: proving that it is possible to underestimate the taste of the American public, the new Lucille Ball program Life with Lucy did very poorly in the ratings despite heavy promotion.

(RvdH) More proof: the Excalibur car company went out of business.

(At this point a large safe-like object wheels into the room, dragging the Telzey and the Mobcomm behind it. This prompts much discussion and commentary.)

(CH) Minicult, life is stranger than comic strips department: Store Owner Beaten With Chili Dog.

(AA) Minicult: someone was giving out flyers at the T station today and I took one. It gives a new explanation for why so many people are taking drugs: it's because of the "cracked philosophies that have invaded our public schools since the sixties, namely, there is no God, your ancestors were monkeys, and you cannot pray in public school."

(JME) Late Minicult: in a case concerning whether or not a comatose fireman would be allowed to die or not, the dissenting justice got up in court and gave a diatribe on what this kind of thing leads to, and he blames it on secular humanism, which he equates with modern paganism.

(AA) Minicult: in the "surrogate mother" case where the woman is claiming the baby is really her husband's: it's not. They did blood tests.

(SLP) Miller motion.

(AA) Second.

It fails.

Somebody comes up with a second Miller motion which passes.

\vspace{12pt}

\noindent
Meeting adjourned, 1749 SST.

\vspace{18pt}

\centerline{Sincerely Sitcomm sub Seventy-Seven,}
\centerline{Jennifer Hawthorne, Onseck}

\end{document}